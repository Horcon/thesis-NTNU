\chapter*{Sammendrag}

Denne rapporten beskrier et forskningsprosjekt som hadde som mål å utforske metoder til å redusere byrden på det utstyret fra høyeste hylle brukt til å trene nevrale nettverk gjennom å bruke mange mindre nevrale nettverk, i stedet for et stort nettverk.
Nevrale nettverk brukt i denne masteroppgaven er re-trent iterativt på et progressivt større sett med data, og deretter brukt i et tre-hierarki.
Forskjellige lærdommer fra tidligere forskning ute i feltet er brukt til å akselerere utviklingen av den nevrale nettverk modellen til å oppnå gode resultater før hoved-eksperimentene begynner.

Gjennom treningsprosessen, en modifisert tapsfunksjon med en filter er brukt til å rettlede det nevrale nettverket til å oppnå bedre klassifiseringer for gitte dataprøver.
Filteret er brukt gjennom å legge til et ekstra lag med nevroner etter softmax laget, som er deretter fjernet etter treningsprosessen er ferdig.
Vektene av dette ekstra-laget er manuelt modifisert til å overføre resultatene fra softmax-laget direkte til ekstra-laget.
Ettersom målet med denne prosessen er å skale bedre klassifiseringer, det resulterende nettverket er fjernet på slutten av denne prosessen.
Bedre klassifiseringer er brukt som grunnlag for grupperingen av datasettet brukt i denne masteroppgaven.

Den iterative re-treningen tar disse bedre klassifiseringene, og bruker de i treningsprosessen ettersom mer og mer av datasettet er bearbeidet.
Til slutt er nettverkene samlet i et tre, og gruppene er kjedet sammen til å distribuere datasettet til mindre fragmenter.
Både den iterative re-treningen og det nevrale nettverk-treet er forsøkt i kombinasjon.
I tillegg er en kontroll-gruppe som ikke bruker den iterative re-treningen brukt til å danne et tre.

Resultatet av masteroppgaven viser at iterativ re-trening har noe effekt på størrelsen og kvaliteten av de opprinnelige data gruppene, så lenge man justerer på parameterne i nettverkene korrekt.
Gjennom trærne generert gjennom iterativ re-trening, vellykkede eksperiment er vist gjennom sammenligning av flere lyd klipp som ble gruppert i samme og forskjellige tre noder.
Modifisering av tapsfunksjonen er vist til å ha liten effekt, men hele prosessen viser en klar hevelse av kvaliteten over alternativene.