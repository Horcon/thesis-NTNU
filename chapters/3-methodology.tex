\chapter{Methodology}

%Long(est) chapter of all probably, 10-20 pages
%Somewhat historical walkthrough through the thesis project, in order of how I did the tasks
%Pre-section: Explain the methodology-experiment chapter division, agile thesis work
Development of the hypothesis for this master thesis and the research questions has begun during the summer of 2019, after contemplating the results of the project work done in the Image Processing and Analysis course.
Over the following year, various parts of the thesis scope and hypothesis have been adjusted to fit in the allocated time, and based on work done during the thesis.
In the autumn semester of 2019, preliminary work was done to assess the feasibility of the master thesis.
Based on the results of these feasibility studies, the spring semester work was focused on implementing the iterative re-training process and tree hierarchy generation.

The research questions of this thesis are aimed at the final result at the end of the development process. 
However, the reasoning behind the decisions taken during development needs to be documented.
The development of the code used in this thesis was done using agile methods.
After a module was developed, it was repeatedly tested, and based on the results of these tests; the next tests were created.

The following chapter presents the methodology of the development done in this thesis.
For each section that has experiments associated with decisions taken during development, the intent behind the experiment is explained.
As the thesis does not seek to answer the question of how to develop iterative re-training process or tree hierarchy generation, the experiments done during the development of this thesis are documented in chapter 4.


%Section 1: Dataset (5+ pages)
%Detailed walkthrough of what I used as my dataset, processing it
%Use the "Stage 1, 2, 3, 4" terminology used in the dataset preparation
%Include results from Specialization in mobile - no, experiment 1
%Refer to experiment 1 in Stage 3, experiment on why MFCC DCT 3 and the others were selected, with md3 being the primary


%Section 2: Neural network used
%Two "main" sections, first section from autumn semester, second from spring semester
%First section define the early "model" built with the help of marine paper
%First section refers to experiment 2, which details on the results for the custom layer
%Second section refers to experiment 2 producing good results, final layer being used
%Selection of the network model done by repeatedly attempting multiple combinations of parameters
%Display a list of parameters tested

%Detail the methodology used in creating the "base" neural network used in this thesis

%Section 3: Loss function
%In detail provide information on how I created my custom loss function filters, tuned the parameters
%Here is NOT the reason I went down from 10 classes to 3
%Explain the logic behind the filter types, show graphs that can illustrate some of this
%Generate graph with gnuplot like in example, more natural?
%Explain the logic behind the filter values, why these in particular

%Q: Provide this info 10->3 in methods or results? IMO methods but can see how it would be relevant in results

%Section 4: Iterative training
%First of the two sort of relevant to the results sections
%Specify the parameters used in determining the correct parameters in developing iterative training
%Specify what loss functions and parameters would used in iterative training depending on what results

%Section 5: Tree generation
%Tree generation, same as iterative training in terms of specifying what I did, results of this goes into results/discussion
%Specify what iterative training parameters will be used depending on the criteria in the results
