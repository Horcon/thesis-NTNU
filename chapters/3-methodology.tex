\chapter{Methodology}

%Long(est) chapter of all probably, 10-20 pages
%Somewhat historical walkthrough through the thesis project, in order of how I did the tasks
%Pre-section: Explain the methodology-experiment chapter division, agile thesis work

%Section 0: Hardware and software used in the project
%Hardware: Specify CPU/GPU/RAM etc used, available total storage (for flex?)
%Software: Specify Python + Tensorflow versions (2.0 early on, 2.1 later, 2.2 available in the end but not used)
%Software python: Include info on Librosa, Scikit and other large libraries used and why

%Section 1: Dataset (5+ pages)
%Detailed walkthrough of what I used as my dataset, processing it
%Use the "Stage 1, 2, 3, 4" terminology used in the dataset preparation
%Include results from Specialization in mobile - no, experiment 1
%Refer to experiment 1 in Stage 3, experiment on why MFCC DCT 3 and the others were selected, with md3 being the primary


%Section 2: Neural network used
%Two "main" sections, first section from autumn semester, second from spring semester
%First section define the early "model" built with the help of marine paper
%First section refers to experiment 2, which details on the results for the custom layer
%Second section refers to experiment 2 producing good results, final layer being used
%Selection of the network model done by repeatedly attempting multiple combinations of parameters
%Display a list of parameters tested

%Detail the methodology used in creating the "base" neural network used in this thesis

%Section 3: Loss function
%In detail provide information on how I created my custom loss function filters, tuned the parameters
%Here is NOT the reason I went down from 10 classes to 3
%Explain the logic behind the filter types, show graphs that can illustrate some of this
%Generate graph with gnuplot like in example, more natural?
%Explain the logic behind the filter values, why these in particular

%Q: Provide this info 10->3 in methods or results? IMO methods but can see how it would be relevant in results

%Section 4: Iterative training
%First of the two sort of relevant to the results sections
%Specify the parameters used in determining the correct parameters in developing iterative training
%Specify what loss functions and parameters would used in iterative training depending on what results

%Section 5: Tree generation
%Tree generation, same as iterative training in terms of specifying what I did, results of this goes into results/discussion
%Specify what iterative training parameters will be used depending on the criteria in the results
